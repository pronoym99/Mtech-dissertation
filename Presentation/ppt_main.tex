\documentclass[9pt]{beamer}
\usepackage[utf8]{inputenc}
\usepackage{textcomp}
\usepackage{graphicx}
\usepackage{ragged2e}
\usepackage{subfig}
\hypersetup{pdfpagemode=FullScreen}
\usetheme{Copenhagen}

% bibliography setup %
\usepackage[backend = biber,sorting = none]{biblatex}
\addbibresource{ppt_references.bib}
\setbeamertemplate{bibliography item}{\insertbiblabel}


%Information to be included in the title page:
\title{Analysis of stock market recommendations using computer vision}
\subtitle{\scriptsize{In partial fulfillment of the requirements of the degree of \\ M.Tech. in Data Science}}

\author[Pronoy Mandal (Reg.no.: 20-27-09)]{\large Pronoy Mandal}
\institute{\normalsize Data Science \\ Department of Applied Mathematics \\ Defence Institute of Advanced Technology, Pune}



\date{\today}
\setbeamertemplate{frame numbering}[fraction]

\newcommand*\oldmacro{}%
\let\oldmacro\insertshorttitle%
\renewcommand*\insertshorttitle{%
  \oldmacro\hfill\hfill\hfill\hfill\hfill\hfill %
  \insertframenumber\,/\,\inserttotalframenumber}


\begin{document}

\begin{frame}[plain]
 \titlepage
\end{frame}

\begin{frame}
\frametitle{What are we going to learn?}
\tableofcontents
\end{frame}

\section{Part I - Data science: the industrial perspective}

\subsection{Need of data science in industries}

\begin{frame}
 \frametitle{Need of data science in industries}
 \textit{\textcolor{blue}{``Why do we need data science when most of our modern devices are computer controlled and heavily automated?"}} \\

 \begin{enumerate}
    \item<2-> Voluminous data generated from user metrics helps us optimise various processes
    \item<2-> Helps personalise Human Machine Interaction (HMI)
    \item<2-> Helps understand data whose volume, velocity and variety etc. can't be processed and understood easily in terms of patterns and trends

 \end{enumerate}
\end{frame}

\subsection{Data Science - Industries vs Consumer driven}

\begin{frame}
 \frametitle{Data Science - Industries vs Consumer driven}
 \textit{\textcolor{blue}{"Which of the data science domains: industries OR consumer driven applications deserves more focus"}}  \\
 \begin{enumerate}
    \item<2-> Large industries produce thousands of terabytes of data as opposed to few gigabytes of data for small industries or home based solutions over a year.

    \begin{block}{Fun fact!!}<3->
        A smart home generates about 1 GB of data per week ($\approx$ 200 MB/day) while a simple processing rack of a manufacturing plant may use up to 1 TB of memory.
    \end{block}

    \item<4-> Industries use wide varied ML algorithms for analysing and acting on collected data as opposed to dedicated algorithms and data collecting regimes for small industries
    \item<4-> Industrial applications of data science have grown exponentially with the advent of Industry 4.0 (IoT - connected devices and machines)
 \end{enumerate}

\end{frame}

\begin{frame}
 \frametitle{Data Science - Industries vs Consumer driven (contd.)}
 \begin{enumerate}
 \setcounter{enumi}{3}
     \item Applications in industries are delay sensitive, while in homes or small consumer driven solutions they are not.
     \item Harsh environments are prominent in factories and industries while the same may not be true for home applications.
     \item Industrial applications are \alert{mission critical} while small scale data science solutions may not pertain to some mission.
 \end{enumerate}

\end{frame}

\subsection{Enabling technologies for data science}

\begin{frame}
 \frametitle{Data acquisition}

 A typical factory or industry employs hundreds of sensors at various places. The data from these sensors need to be efficiently communicated to processing or storage hubs.\cite{Dai_2019} \\

 In general (and unfortunately), \alert{increased data rates usually correspond to decreased data range}.

 \only<2>{ \begin{figure}
     \centering
     \includegraphics[scale=0.5]{enabling_tech_data_acquire.PNG}
     \caption{Bandwidth vs Coverage for a variety of technologies}
     \label{fig:data_acq}
 \end{figure}}


\end{frame}

\begin{frame}
 \frametitle{Data preprocessing, storage and communications}

 \begin{itemize}
     \item<1-> Data preprocessing - Data consists of heterogeneous data points which could be erroneous and noisy as well as redundant.

     \only<2>{ \begin{figure}
     \centering
     \includegraphics[scale=0.35]{data_preprocessing.PNG}
     \caption{Data preprocessing techniques}
     \label{fig:data_prepro}
     \end{figure}}


     \item<3-> Storage - It consists of a \textit{network} of interconnected storage devices or platforms. Almost all common DBMS software is an ideal choice for data storage.

     \only<4>{ \begin{figure}
     \centering
     \includegraphics[scale=0.3]{database_logos.PNG}
     \caption{Commonly used DBMS software\cite{dbms_pic}}
     \label{fig:dbms_logos}
     \end{figure}}

 \end{itemize}

\end{frame}

\begin{frame}
 \frametitle{Data analytics}

 \begin{itemize}
     \item<1-> Data analytics - To analyse processed data (so as to take subsequent actions on the same) using any of the methods - 1) Statistical modeling schemes, 2) Data mining schemes, 3) Machine learning schemes and 4) Data visualization. \\

     \only<2->{Luckily, business value increases with increasing complexity of an approach.}

     \only<3>{ \begin{figure}
     \centering
     \includegraphics[scale=0.35]{enabling_tech_data_anal.PNG}
     \caption{Business value vs complexity for various data analysis approaches}
     \label{fig:data_anal}
     \end{figure}}

 \end{itemize}

\end{frame}

\begin{frame}[allowframebreaks]{References}

\printbibliography

\end{frame}

\begin{frame}{}
  \centering \Huge
  \emph{Thank You}
\end{frame}

\end{document}
