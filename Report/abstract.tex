\chapter*{Abstract}

There are a large number of business NEWS channels broadcasting a variety of topics throughout morning and evening primetime. These range from simple stock market recommendations (strategies to BUY/SELL/HOLD) to detailed fundamental and technical analyses of various companies to in-depth quarterly performance reviews of various parties in the market. With the availability of such a volume of information from public sources; it becomes difficult to assimilate all the information in a single place and analyse it for meaningful information. \par

This project encompasses the ability to analyse NEWS which gives stock market recommendations (a subset of business NEWS) and presents the results in a tabular format to the user for analysis. The tabular format includes basic information from the broadcast such as the telecast date of the show, the analyst presenting the same, the stock which is being discussed, the NSE listing symbol of the concerned stock, the recommendation etc. (totalling 8 parameters). The exact reason why the Securities and Exchange Board of India (SEBI) is attempting to analyse NEWS videos is something that can’t be specified within the scope of this report due to its confidentiality. However, it can be said that this project’s output forms the base as well as the input of many anomaly (in the context of violations in the securities market) detection models at a later stage. \par

This project goes a step ahead to ease and automate the job of SEBI officers from the surveillance department by deploying the same onto an MLOPs platform and creating an equivalent free-running workflow in a virtual environment. \par

The aforementioned reason forms the principal motivation for doing this project i.e. it helps automate a large part of a task that is prone to human errors and mistakes such as collecting data manually by watching the NEWS shows and at the same time reduces the time required to carry out such tasks. Apart from that, since an MLOPs workflow is included in the project, this project helps establish various principles which need to be followed for any ML project in general so that it can be deployed into production.
